\documentclass[a4paper,12pt]{article}
\usepackage{amssymb, algorithm, fancyhdr, algorithmicx, algpseudocode}

\newcommand{\m}{\mathbb}
\newcommand{\B}{\mathcal}

\newtheorem{name}{Printed output}[section]
\newtheorem{mydef}{Definition}[section]
\newtheorem{mylem}{Lemma}[section]
\newtheorem{mythm}{Theorem}[section]
\newtheorem{mycor}{Corollary}[section]

\pagestyle{fancy}
\setlength{\parindent}{0mm}
\setlength{\headheight}{15.0pt}
\lhead{Angus Gruen, u5561004}
\rhead{Defining an Model of Computation}

\begin{document}

	An algorithm is a finite length list of computations. 

	\vspace{2mm}

	A model of computation is a list of computations that an algorithm is allowed to use.

	\vspace{2mm}

	Some simple examples of models of computations are:
	\textbf{
	\begin{enumerate}
		\item Inputs
		\item If statments
		\item For Loops
		\item Memory Assignment
	\end{enumerate}}
	This is essesntially a turing machine. If we dropped the ability for Memory Assignment, we get a weaker model known as a Finite State Machine.
	\textbf{
	\begin{enumerate}
		\item Inputs
		\item If statments
		\item For Loops
	\end{enumerate}}
	A interesting example of a model of computation is straight edge geometry defined by the following list of allowed computions 
	\textbf{
	\begin{enumerate}
		\item You start with 2 points length 1 apart.
		\item You can draw a straight line between any 2 points.
		\item Given 2 points you can draw the circle centered at one point and passing through another.
		\item You can add a point at an intersection of any two points.
		\item You can extend any straight line into a ray.
		\item You can output the distance between any two points along any line drawn between the points. 
	\end{enumerate}}

\end{document}